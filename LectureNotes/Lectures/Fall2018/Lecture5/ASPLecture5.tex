% ==============================================================================



%% the first is for standalone compiling,
%%   the second looks for the master document

%\documentclass[classnotes]{fillsntsx}
%\documentclass[../../../Master/AppliedStochastics.tex]{subfiles}
\documentclass{article}
\usepackage{fullpage}

%%
% Add your macros here; they'll be included in pdf and html output.
%%

\usepackage{amsmath,amssymb,amsthm}
\usepackage{mathtools} %For arrows with text above and below%

\newcommand{\Z}{\mathbb{Z}}    % integers
\newcommand{\N}{\mathbb{N}}    % naturals
\newcommand{\R}{\mathbb{R}}    % reals

\newcommand{\E}{\mathbb{E}}    % expectation
\renewcommand{\P}{\mathbb{P}}  % probability
\DeclareMathOperator{\sd}{sd}
\DeclareMathOperator{\var}{var}
\DeclareMathOperator{\cov}{cov}

% distributions
\DeclareMathOperator{\Normal}{Normal}
\DeclareMathOperator{\Poisson}{Poisson}
\DeclareMathOperator{\Beta}{Beta}
\DeclareMathOperator{\Binom}{Binomial}
\DeclareMathOperator{\Gam}{Gamma}
\DeclareMathOperator{\Exp}{Exponential}
\DeclareMathOperator{\Cauchy}{Cauchy}
\DeclareMathOperator{\Unif}{Unif}
\DeclareMathOperator{\Dirichlet}{Dirichlet}

\newcommand{\given}{\;\vert\;}

\theoremstyle{definition}
\newtheorem{thm}{Theorem}[section]
\newtheorem{prop}[thm]{Proposition}
\newtheorem{coro}[thm]{Corollary}
\newtheorem{lemma}[thm]{Lemma}
\newtheorem{defn}[thm]{Definition}
\newtheorem{exmp}[thm]{Example}
\newtheorem{rmk}[thm]{Remark}
\newtheorem{exer}[thm]{Exercise}
\newtheorem{nota}[thm]{Notation}
\newtheorem*{note}{Note}
\newtheorem*{sol}{Solution}

\usepackage{graphicx} %For including pictures from outside files%


%% without the master file this macro does nothing
%\course{Applied Stochastic Processes}


\author{Nathan}  % your name
\date{October 12 and 15}    % the date of the lecture

% ==============================================================================
%
\begin{document}
%
% ==============================================================================

\makelecture

\textbf{Homework Problem:} Spiders live at the points of a PPP on a long wire with uniform intensity of 0.1 per centimeter. Each claims the area withint 2 cm on either side, so a fly landing on the wire is attacked by all spiders within 2 cm of it.

\begin{enumerate}
\item A fly lands on the wire independent of the spiders. What is the probability that the fly is safe?

\item We have a PPP of flies (independent of the spiders) with uniform intensity 0.1 per cm landing on a 1 meter section of wire. Let $F_k=\#\,\{\textrm{flies attacked by $k$ spiders}\}$. What is $\mathbb{P}\{F_k=n\}$?

\item \emph{Bonus:} compute the mean and variance of the total length of wire in a 1 meter section \emph{not} claimed by spiders.
\end{enumerate}

\hrulefill\\\\

\textbf{More rain:} a hailstorm falls on our patio for 1 hour. The rate of hailstones having mass $x$ grams is $\dfrac{1}{x}e^{-x}$ stones per hour (in total on the patio). Assume the hail is pure ice (water has density 1 gram per cubic cm).

\begin{enumerate}
\item How many hailstones have mass greater than 0.01 grams fall?

\item How many hailstones total fall?

\item Pick a random stone with mass greater than 1 gram. What is the chance it weighs more than 2 grams?

\item What is the total weight of all hails?

\item Line up the hailstones side-by-side. What is the total length of hail?
\end{enumerate}

\textbf{1).} 
Let $N([a,b])=\#\,\{\textrm{hailstones with mass $a\leq x\leq b$}$. 
Then $N\sim PPP(\mu)$ on $(0,\infty)$ with $d\mu(x)=\dfrac{1}{x}e^{-x}dx$. 
Thus
$$N([0.01,\infty))\sim\textrm{Pois}\left( \int_{0.01}^\infty \frac{1}{x}e^{-x}dx\right).$$

Define $E(x)=\int_x^\infty\frac{1}{y}e^{-y}dy$. 
Then $\mathbb{E}[N([0.01,\infty))]=E(0.01)=4.03793$ 
(computed in R, using the gsl package)
and $\textrm{var}[N([0.01,\infty))]=E(0.01)=4.03793$.

\textbf{2).} 
How many hailstones total fall? 
$\mathbb{E}[N([\epsilon,\infty))]=E(\epsilon)\xrightarrow{\epsilon\searrow0}\infty$ so $\mathbb{P}\{N(0,\infty))=\infty\}=1$. 
Picture: with the patio as the horizontal axis and $x$ as the vertical axis, there's high density of low $x$ values along the patio.

\paragraph{Property: (conditional uniformity)} 
Let $N\sim PPP(\mu)$ on $X$ and $A\subset X$ with $\mu(A)<\infty$. 
Conditioned on $N(A)$, the points of $N$ that fall in $A$ are i.i.d. with distribution proportional to $\mu$; 
i.e., if $B\subset A$ then 
$$\mathbb{P}\{N(B)=k | N(A)=n\} = {{n}\choose{k}} \left(\frac{\mu(B)}{\mu(A)}\right)^k\left( 1-\frac{\mu(B)}{\mu(A)}\right)^{n-k}.$$

\textbf{3).} 
Let $B=[2,\infty)$ and $A=[1,\infty)$. 
The mass of stones with mass greater than 1 gram has probability density $\dfrac{\frac{1}{x}e^{-x}}{E(1)}$ for $x\geq 1$. 
So $$\mathbb{P}\{\textrm{stone}>2g|\textrm{stone}>1g\}=\dfrac{E(2)}{E(1)}=0.2228992.$$\\

\textbf{4).} 
Let $W=\int_0^\infty x\,dN(x)=\sum_ix_i=\textrm{total wieght of all hailstones}$. 
We find the mean and variance of $W$. 
$$\mathbb{E}[W]=\int_0^\infty x\frac{1}{x}e^{-x}\,dx=1\,\textrm{gram}$$
since $\mathbb{E}[\int f\,dN]=\int f\,d\mu$.\\

\textbf{Lemma:} $\textrm{var}[\int f(x)\,dN(x)]=\int f(x)^2\,d\mu(x)$.

\textit{Proof:} 
Let $C_j^{(\epsilon)}$ be a partition of $X$ with all $C_j^{(\epsilon)}$ be ``$\epsilon$-small", 
and let $z_j^{(\epsilon)}\in C_j^{(\epsilon)}$ be the center of each $C_j^{(\epsilon)}$. Then
$$\int f(x)\,dN(x)=\sum_if(x_i)\approx\sum_j N(C_j^{(\epsilon)})f(z_j^{(\epsilon)})$$
so 
$$\begin{split}
    \textrm{var}\left[  \int f(x)\,dN(x) \right] &\approx \textrm{var}\left[\sum_j N(C_j^{(\epsilon)})f(z_j^{(\epsilon)}) \right]\\
        &=\sum\textrm{var}[ N(C_j^{(\epsilon)})f(z_j^{(\epsilon)})]\\
        &=\sum f(z_j^{(\epsilon)})^2\mu(C_j^{(\epsilon)})\\
        &\xrightarrow{\epsilon\searrow0}\int f(x)^2d\mu(x)   \qquad \blacksquare
\end{split}$$

Thus $\textrm{var}[W]=\int_0^\infty x^2\frac{1}{x}e^{-x}\,dx=1$.

\textbf{5).} Since the density of water is 1 g/$\textrm{cm}^3$, $\mathbb{E}[\textrm{length}]=\int_0^\infty x^{1/3}\frac{1}{x}e^{-x}\,dx=\Gamma(\frac{1}{3})$.

\hrulefill\\\\

\textbf{Fact:} if $X$ and $Y$ are random variables, $\mathbb{E}[e^{i\alpha X}]=\mathbb{E}[e^{i\alpha Y}]$ $\forall \alpha\in\mathbb{R}$ if and only if $X\overset{d}{=}Y$ (i.e. $\mathbb{P}\{X\in A\}=\mathbb{P}\{Y\in A\}$ for all $A$, or $\mathbb{E}[f(X)]=\mathbb{E}[f(Y)]$ for all $f$).\\

\paragraph{Characteristic functions, in general}
The \emph{characteristic function} for $X$ is $\phi_X(\alpha)=\mathbb{E}[e^{i\alpha X}]$; in other words, $\phi_X$ is the Fourier transform of the density function $f_X$ of $X$.
Characteristic functions are convenient for calculating moments, but \emph{not} probabilities.

\textbf{Lemma:} Let $\psi (\alpha)=\log\phi_X(\alpha)$ be the \emph{cumulant generating function}. 
Then 
$$\frac{d}{d\alpha}\psi(0)=i\,\mathbb{E}[X]$$
$$\frac{d^2}{d\alpha^2}\psi(0)=-\textrm{var}[X]$$

\textit{Proof:} \emph{(1)} 
$$\begin{split}
    \frac{d}{d\alpha}\mathbb{E}[e^{i\alpha X}]&=\mathbb{E}\left[\frac{d}{d\alpha}e^{i\alpha X}\right]\\
        &=\mathbb{E}[iXe^{i\alpha X}]
\end{split}$$
so $\frac{d}{d\alpha}\psi(0)=\dfrac{\mathbb{E}[iXe^{i0 X}]}{\mathbb{E}[e^{i0 X}]}=i\mathbb{E}[X]$.

\emph{(2)} Similarly:
$$\frac{d^2}{d\alpha^2}\psi(\alpha)=\frac{ \mathbb{E}[-X^2e^{i\alpha X}]-\mathbb{E}[iXe^{i\alpha X}]^2   }{\mathbb{E}[ie^{i\alpha X}]^2}$$
$$\frac{d^2}{d\alpha^2}\psi(0)
    =\mathbb{E}[X^2]-\mathbb{E}[X]^2
    =\textrm{var}[X]$$
as desired. $\blacksquare$
\newline



\paragraph{Characteristic functions for additive functionals of Poisson point processes:} 
Let $N\sim PPP$ on $X$ with intensity $\mu$ and let $f:X\rightarrow\mathbb{R}$. 
Then $$\mathbb{E}\left[\exp\left(i\alpha\int f(x)\,dN(x)\right)\right]=\exp\left(\int_X (e^{i\alpha f(x)}-1)\mu(dx) \right).$$

\textit{Lemma:} If $Z\sim\textrm{Pois}(\gamma)$, then 
$$\mathbb{E}[e^{i\alpha Z}]=\sum_{n\geq0}e^{i\alpha n}e^{-\gamma}\frac{\gamma^n}{n!}=e^{-\gamma}\sum_{n\geq 0}(\gamma e^{i\alpha})^n/n!=\exp(\gamma(e^{i\alpha}-1))$$


\textit{Proof (theorem):} Let $f$ be piecewise constant $f(x)=f_i$ for $x\in A_i$ with $\sqcup_i A_i=X$. Then $\int f(x)\,dN(x)=\sum_i N(A_i)f_i$ (and the $N(A_i)$ are all independent) so 
$$\begin{split}
\mathbb{E}[e^{i\alpha \int f\,dN}]&=\mathbb{E}\left[\prod_je^{i\alpha N(A_j)f_j}\right]\\
&=\prod_j\mathbb{E}[e^{i\alpha N(A_j)f_j}]\\
&=\prod_j\exp(\mu(A_j)(e^{i\alpha f_j}-1))\\
&=\exp\left( \int_X(e^{i\alpha f(x)}-1)\mu(dx) \right)  \qquad \blacksquare
\end{split}$$


\emph{Corollary:} $\mathbb{E}[\int f\,dN]=\int f\,d\mu$ and $\textrm{var}[\int f\,dN]=\int f^2\,d\mu$.

\paragraph{The Cauchy process:} 
Let $N\sim PPP$ on $[0,\infty)\times(\mathbb{R}\backslash\{0\})$ with mean measure $\mu(dt,dx)=\dfrac{dt\,dx}{|x|^2}$.
Picture: with $t$ as the horizontal and $x$ as the vertical axes, infinite density of points approaching near the $t$-axis.

Let $C_t=\int_0^t\int_\mathbb{R} x\,dN(s,x)=\textrm{sum of $x$-coordinates of points in $[0,t]\times\mathbb{R}$}$. 

Note: 
$$\begin{split}
    \mathbb{P}\{\textrm{no jumps in $[t,t+\epsilon)$}&=\mathbb{P}\{C_s=C_t:s\in[t,t+\epsilon)\}\\
        &=\mathbb{P}\{N([t,t+\epsilon)\times\mathbb{R})=0\}\\
        &=\exp\left(-\epsilon\int\frac{1}{|x|^2}dx\right)
        &=0
\end{split}$$

Also:
$$\mathbb{P}\{\textrm{no jumps bigger than $\delta$ in $[t,t+\epsilon)$}\}=\exp\left(-2\epsilon\int_\delta^\infty\frac{dx}{x^2}\right)=e^{-2\epsilon/\delta}.$$


What is the distribution of $C_t$?

$$\begin{split}
    \mathbb{E}[e^{i\alpha C_t}]&=\exp\left( \int_0^t\int_{-\infty}^\infty(e^{i\alpha X}-1)\frac{1}{|x|^2}dx\,dt  \right)\\
        &=\exp(-t|\alpha|)\\
        &=\int_{-\infty}^\infty e^{iz\alpha}\frac{dz}{\pi t(1+(z/t)^2)}
\end{split}$$
i.e. $C_t\sim\textrm{Cauchy}(t)=\textrm{probability density}\dfrac{1}{\pi t(1+(z/t)^2)}$.

An interesting property: $C_n=C_1+(C_2-C_1)+...+(C_n-C_{n-1})=n\textrm{ i.i.d. }\sim C_1$. Thus $\frac{1}{n}C_n\overset{d}{=}C_1$ and $\textrm{var}[\frac{1}{n}C_n]=\frac{1}{n}\textrm{var}[C_1]$.

% ==============================================================================
%
\end{document}
%
% ==============================================================================
